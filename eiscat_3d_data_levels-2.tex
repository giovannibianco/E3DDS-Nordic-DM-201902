\begin{frame}[fragile,t]
% \frametitle{\hfill}
  \MyHeading{
    % \color{white}
    \EC Data Levels}
\vspace{\mytopbit}
\tiny
\centering
\begin{tabular}{l|l l l c r}
{Level} & {Type}        & {Produced by} &  {Storage } & {Format} & {Rate}\\ \hline
% \bf 0  & Sampled antenna        & Antenna & None & Binary \\
%        & voltages               & sub-arrays & & \\
\bf 1a & Ring buffer data                & $1^{\rm st}$ stage beam & 4 months$^*$ & UDP stream/ & {\colblue \bf $\leq 4$~Tb/s~\footnote{At full 52 Msamples/s from each FSRU}} \\
       &                                 & former &  & HDF5  \\
\bf 1b & Beam-formed data                & $2^{\rm nd}$ stage beam & 4 months$^*$ & HDF5 & {\colblue \bf 64~Gb/s} \\
       &                                 & former & &  \\
\bf 2  & Time integrated                 & All sites & Archived & \HDF \\
       & correlated data & & & \\
\bf 3a & Physical parameters             & All sites & Archived & \HDF &  \\
\bf 3b & 3D-voxel parameters             & Operations centre & Archived & \HDF & {\colblue \bf $\approx 1$~Gb/s} \\
\bf 4  & Derived geophysical & Users & Users & Publications etc \\
       & parameters & & & 
\end{tabular}

\small
\bitm
\item {The \ED \DCs will receive, serve and archive all data at levels 2 and 3.}
  \item { Data used in research should be given Persistent Identifiers (PIDs) according to a common standard such as DOI, DataCite, or similar, to be unambiguously citable in publications.}
\item {A 4~months period is selected as this is the estimated time required to perform a ``real-time'' analysis on low-level data.}
\item {A portion of the level~1 data will also be archived permanently, on the order of $1\%$ of the level~1 data rate, e.g. one beam per site and/or bandwidth-limited data.}
  \eitm
  
\end{frame}
